\documentclass[a4paper]{book}
\usepackage[times,inconsolata,hyper]{Rd}
\usepackage{makeidx}
\usepackage[utf8]{inputenc} % @SET ENCODING@
% \usepackage{graphicx} % @USE GRAPHICX@
\makeindex{}
\begin{document}
\chapter*{}
\begin{center}
{\textbf{\huge Package}}
\par\bigskip{\large \today}
\end{center}
\begin{description}
\raggedright{}
\inputencoding{utf8}
\item[Type]\AsIs{Package}
\item[Title]\AsIs{Difference between Varying Distributions Test (DVDtest)}
\item[Version]\AsIs{0.1}
\item[Date]\AsIs{2019-03-24}
\item[Author]\AsIs{Meng Xu, Philip Reiss}
\item[Maintainer]\AsIs{Meng Xu }\email{mxu@campus.haifa.ac.il}\AsIs{}
\item[Description]\AsIs{See DVDtest.}
\item[RoxygenNote]\AsIs{6.1.1}
\item[License]\AsIs{GPL (>= 2)}
\item[Encoding]\AsIs{UTF-8}
\item[Imports]\AsIs{gamlss, mgcv, ggplot2, reshape2, parallel, gamlss.dist}
\end{description}
\Rdcontents{\R{} topics documented:}
\inputencoding{utf8}
\HeaderA{DVDtest-package}{Difference between Varying Distributions Test (DVDtest)}{DVDtest.Rdash.package}
%
\begin{Description}\relax
This package contains functions of a test on the difference between varying distributions.
\end{Description}
%
\begin{Author}\relax
Meng Xu, Philip Reiss
\end{Author}
%
\begin{References}\relax
reiss-EMR18.pdf
\end{References}
%
\begin{SeeAlso}\relax
\code{\LinkA{DVDtest}{DVDtest}}
\end{SeeAlso}
\inputencoding{utf8}
\HeaderA{DVDtest}{Difference between Varying Distributions Test (DVDtest)}{DVDtest}
\keyword{minp}{DVDtest}
\keyword{permutation}{DVDtest}
%
\begin{Description}\relax
Testing the difference of two varying distributions.
\end{Description}
%
\begin{Usage}
\begin{verbatim}
DVDtest(ydata1, ydata2, nperm, eval.index.grid, dist.method = "wass",
  mgcv.gam = TRUE, ..., exclude = NULL, permadj = FALSE,
  mc.cores = 1)
\end{verbatim}
\end{Usage}
%
\begin{Arguments}
\begin{ldescription}
\item[\code{ydata1}] a \code{data.frame} or a \code{list} of \code{data.frame} containing 
at least 3 columns called '\code{.obs}','\code{.index}','\code{.value}' which 
specify which curve the point belongs to (\code{.obs}) at which ('\code{.index}') 
it was observed and the observed value (\code{'.value'}). Other variables are 
available for modelling the varying distribution as well.

\item[\code{ydata2}] same type as \code{ydata1}. If the type of \code{ydata1} and \code{ydata2}
is a \code{list} of \code{data.frame}, the lenghs of two lists must be the same.

\item[\code{nperm}] a scalar, number of permutation

\item[\code{eval.index.grid}] a vector, evaluation grid of \code{.index}

\item[\code{dist.method}] the methods, Wasserstein(\code{'wass'}), L2(\code{L2}), 
L1(\code{'L1'}) and Hellinger(\code{'Hellinger'}) to calculate the distances between 
distributions. Defaults to \code{'wass'}.

\item[\code{mgcv.gam}] a logical variable, whether to apply \code{mgcv::gam} for eastimating 
distributions, whose parameters are a smooth function of a continuous variable. If 
\code{FALSE}, \code{gamlss::gamlss} is adopted.

\item[\code{...}] If \code{mgcv.gam = TRUE}, \code{...} should include \code{formula} and 
\code{family}(=\code{gaulss()}) and other arguments in \code{mgcv::gam}. Otherwise, 
the arguments in gamlss::gamlss are included.

\item[\code{exclude}] works only in the case of \code{mgcv.gam = TRUE}, to exclude the random effect

\item[\code{permadj}] a logical variable, whether to adjust the permutated data to cover the entire range,
esp. working in case of sparsity. Defaults to \code{FALSE}

\item[\code{mc.cores}] the same argument in \code{mclapply} (not work on Windows). Defaults to 1.
\end{ldescription}
\end{Arguments}
%
\begin{Details}\relax
This is the Details section
\end{Details}
%
\begin{Value}
.indexa vector, evaluation grid 
\end{Value}
%
\begin{Note}\relax
\begin{itemize}

\item If \code{mgcv.gam} is TRUE and \code{exclue} is NULL (default settings), 
then \code{formula. <- list(.value\textasciitilde{}s(.index)+s(.obs, bs="re"), \textasciitilde{}s(.index))} and 
\code{exclude. <- "s(.obs)"}

\end{itemize}

\end{Note}
%
\begin{Author}\relax
Meng Xu, Philip Reiss
\end{Author}
%
\begin{References}\relax
reiss-EMR18.pdf
\end{References}
%
\begin{Examples}
\begin{ExampleCode}

 p=6
 mu1<-function(t) 0.2*(p-1)*sin(pi*t)+t+1
 mu2<-function(t) -0.2*(p-1)*sin(pi*t)+t+1
 sig1 <- function(t) t+1
 sig2 <- sig1
 nperson=10
 fun1<-function(t) rnorm(nperson,mu1(t),sig1(t))
 fun2<-function(t) rnorm(nperson,mu2(t),sig2(t))
 tp<-seq(0,1,,10)
 data1<-sapply(tp,fun1)
 data2<-sapply(tp,fun2)
 
 library(reshape2)
 colnames(data1)<-tp
 dg1<-melt(data1)
 colnames(dg1)<-c('.obs','.index','.value')
 dg1$.obs<-as.factor(dg1$.obs)
 
 colnames(data2)<-tp
 rownames(data2)<-1:nperson+2*nperson
 dg2<-melt(data2)
 colnames(dg2)<-c('.obs','.index','.value')
 dg2$.obs<-as.factor(dg2$.obs)
 # library(ggplot2)
 # ggplot()+geom_line(data=dg1,aes(x=.index,y=.value,col=factor(.obs)))+
 #   geom_line(data=dg2,aes(x=.index,y=.value,col=factor(.obs)))

 ngrid=50
 ev.grid <- seq(0, 1, , ngrid)
 nperm. <- 50
 
 simu.test<-DVDtest(dg1,dg2,nperm.,ev.grid)
 
 #ggplot(data.frame(simu.test),aes(x=.index,y=pval))+geom_line()+
 # geom_hline(yintercept=0.05,linetype=2,col="red")
\end{ExampleCode}
\end{Examples}
\inputencoding{utf8}
\HeaderA{get.params}{Fitting the generalized gamma (Pearson type III) distribution for each k}{get.params}
%
\begin{Description}\relax
Fitting by the permuted-data distances
\end{Description}
%
\begin{Usage}
\begin{verbatim}
get.params(k, nperm, permarray, eval.index.grid)
\end{verbatim}
\end{Usage}
%
\begin{Arguments}
\begin{ldescription}
\item[\code{k}] a scalar, \code{k}th data.frame in \code{ydata1\&2}

\item[\code{nperm}] a scalar, number of permutation

\item[\code{permarray}] an array, permuted-data distances from \code{wass\_perm}

\item[\code{eval.index.grid}] a vector, evaluation grid of \code{.index}
\end{ldescription}
\end{Arguments}
%
\begin{Value}
\begin{ldescription}
\item[\code{mu}] a vector of mean\end{ldescription}
, \begin{ldescription}
\item[\code{sigma}] a vector of sigma\end{ldescription}
, 
\begin{ldescription}
\item[\code{nu}] a vector of nu\end{ldescription}
, \begin{ldescription}
\item[\code{aic}] the Akaike information criterion in \code{gamlss}
\end{ldescription}
\end{Value}
%
\begin{Author}\relax
Philip Reiss, Meng Xu
\end{Author}
%
\begin{SeeAlso}\relax
\code{\LinkA{get\_params}{get.Rul.params}}
\end{SeeAlso}
\inputencoding{utf8}
\HeaderA{get.pval}{Calculate the corrected P values}{get.pval}
%
\begin{Description}\relax
Calculate the corrected P values
\end{Description}
%
\begin{Usage}
\begin{verbatim}
get.pval(permarray, param.array, realdists, nroi, eval.index.grid, nperm)
\end{verbatim}
\end{Usage}
%
\begin{Arguments}
\begin{ldescription}
\item[\code{permarray}] an array, permuted-data distances from \code{wass\_perm}

\item[\code{param.array}] an array, permuted-data distances from \code{get\_params}

\item[\code{realdists}] a matrix or vector, the value from \code{get\_realdist}

\item[\code{nroi}] a scalar, the length of \code{ydata1}

\item[\code{eval.index.grid}] a vector, evaluation grid of \code{.index}

\item[\code{nperm}] a scalar, number of permutation
\end{ldescription}
\end{Arguments}
%
\begin{Value}
a vector or matrix of p value
\end{Value}
%
\begin{Author}\relax
Philip Reiss, Meng Xu
\end{Author}
%
\begin{SeeAlso}\relax
\code{DVDtest}
\end{SeeAlso}
\inputencoding{utf8}
\HeaderA{get.realdist}{Calculating the distances under the null hypothesis for each roi}{get.realdist}
%
\begin{Description}\relax
Calculating the distances under the null hypothesis for each roi
\end{Description}
%
\begin{Usage}
\begin{verbatim}
get.realdist(k, vdFun, ydata1, ydata2, ind.grid, ..., excl, dist.method)
\end{verbatim}
\end{Usage}
%
\begin{Arguments}
\begin{ldescription}
\item[\code{k}] a scalar, \code{k}th data.frame in \code{ydata1\&2}

\item[\code{vdFun}] a function, \code{gam} or \code{gamlss}, for fitting the varying distributions

\item[\code{ydata1}] see \code{DVDtest}

\item[\code{ydata2}] see \code{DVDtest}

\item[\code{ind.grid}] see \code{eval.index.grid} in \code{DVDtest}

\item[\code{...}] arguments of \code{vdFun}

\item[\code{excl}] an argument of \code{predict} for \code{gam}

\item[\code{dist.method}] see \code{DVDtest}
\end{ldescription}
\end{Arguments}
%
\begin{Value}
a vector of the distances
\end{Value}
%
\begin{Author}\relax
Philip Reiss, Meng Xu
\end{Author}
%
\begin{SeeAlso}\relax
\code{DVDtest}
\end{SeeAlso}
\inputencoding{utf8}
\HeaderA{get\_params}{Fitting the generalized gamma (Pearson type III) distribution}{get.Rul.params}
%
\begin{Description}\relax
Fitting the generalized gamma (Pearson type III) distribution
\end{Description}
%
\begin{Usage}
\begin{verbatim}
get_params(nroi, nperm, permarray, eval.index.grid)
\end{verbatim}
\end{Usage}
%
\begin{Arguments}
\begin{ldescription}
\item[\code{nroi}] a scalar, the length of \code{ydata1} or \code{ydata2}

\item[\code{nperm}] a scalar, number of permutation

\item[\code{permarray}] an array, permuted-data distances from \code{wass\_perm}

\item[\code{eval.index.grid}] a vector, evaluation grid of \code{.index}
\end{ldescription}
\end{Arguments}
%
\begin{Value}
an array, \code{param.array}
\end{Value}
%
\begin{Author}\relax
Meng Xu, Philip Reiss
\end{Author}
%
\begin{SeeAlso}\relax
\code{DVDtest}
\end{SeeAlso}
\inputencoding{utf8}
\HeaderA{get\_realdist}{Calculating the distances under the null hypothesis}{get.Rul.realdist}
%
\begin{Description}\relax
Calculating the distances under the null hypothesis
\end{Description}
%
\begin{Usage}
\begin{verbatim}
get_realdist(vdFun, ydata1, ydata2, ind.grid, ..., excl, mc.cores,
  dist.method)
\end{verbatim}
\end{Usage}
%
\begin{Arguments}
\begin{ldescription}
\item[\code{vdFun}] a function, \code{gam} or \code{gamlss}, for fitting the varying distributions

\item[\code{ydata1}] see \code{DVDtest}

\item[\code{ydata2}] see \code{DVDtest}

\item[\code{ind.grid}] see \code{eval.index.grid} in \code{DVDtest}

\item[\code{...}] arguments of \code{vdFun}

\item[\code{excl}] an argument of \code{predict} for \code{gam}

\item[\code{mc.cores}] a scalar, an argument in \code{mclapply}

\item[\code{dist.method}] see \code{DVDtest}
\end{ldescription}
\end{Arguments}
%
\begin{Value}
a vector or matrix of the distances
\end{Value}
%
\begin{Author}\relax
Meng Xu, Philip Reiss
\end{Author}
%
\begin{SeeAlso}\relax
\code{DVDtest}
\end{SeeAlso}
\inputencoding{utf8}
\HeaderA{make.perms}{Making permutated index}{make.perms}
%
\begin{Description}\relax
Making permutated index
\end{Description}
%
\begin{Usage}
\begin{verbatim}
make.perms(dat1, dat2, nperm, .index, adj)
\end{verbatim}
\end{Usage}
%
\begin{Arguments}
\begin{ldescription}
\item[\code{dat1}] an element of \code{ydata1}

\item[\code{dat2}] an element of \code{ydata2}

\item[\code{nperm}] a scalar, number of permutation

\item[\code{.index}] see \code{eval.index.grid} in \code{DVDtest}

\item[\code{adj}] see \code{permadj} in \code{DVDtest}
\end{ldescription}
\end{Arguments}
%
\begin{Value}
a matrix, permuted indices
\end{Value}
%
\begin{Author}\relax
Philip Reiss, Meng Xu
\end{Author}
%
\begin{SeeAlso}\relax
\code{DVDtest}
\end{SeeAlso}
\inputencoding{utf8}
\HeaderA{multiwass}{Calculating the distances between two gam/gamlss objects}{multiwass}
%
\begin{Description}\relax
Calculating the distances between two gam/gamlss objects
\end{Description}
%
\begin{Usage}
\begin{verbatim}
multiwass(obj1, obj2, newdata1, newdata2, dist.method, ...)
\end{verbatim}
\end{Usage}
%
\begin{Arguments}
\begin{ldescription}
\item[\code{obj1}] a gam/gamlss object

\item[\code{obj2}] another gam/gamlss object

\item[\code{newdata1}] related evaluation grids

\item[\code{newdata2}] related evaluation grids

\item[\code{dist.method}] see \code{DVDtest}

\item[\code{...}] partial arguments in \code{predict}
\end{ldescription}
\end{Arguments}
%
\begin{Value}
a vector, distances
\end{Value}
%
\begin{Author}\relax
Philip Reiss, Meng Xu
\end{Author}
\inputencoding{utf8}
\HeaderA{params2qfunc}{Getting the quantile/density function via parameters}{params2qfunc}
%
\begin{Description}\relax
Getting the quantile/density function via parameters
\end{Description}
%
\begin{Usage}
\begin{verbatim}
params2qfunc(params, family, dist.method)
\end{verbatim}
\end{Usage}
%
\begin{Arguments}
\begin{ldescription}
\item[\code{params}] a vector, parameters for certain distributions

\item[\code{family}] a specific distribution from \code{gam} or \code{gamlss}

\item[\code{dist.method}] see \code{DVDtest}
\end{ldescription}
\end{Arguments}
%
\begin{Value}
quantile or density functions
\end{Value}
%
\begin{Author}\relax
Meng Xu, Philip Reiss
\end{Author}
\inputencoding{utf8}
\HeaderA{qfuncs2wass2}{Distance functions}{qfuncs2wass2}
%
\begin{Description}\relax
Distance functions
\end{Description}
%
\begin{Usage}
\begin{verbatim}
qfuncs2wass2(qfunc1, qfunc2, dist.method = dist.method, ...)
\end{verbatim}
\end{Usage}
%
\begin{Arguments}
\begin{ldescription}
\item[\code{qfunc1}] quantile or density functions from \code{params2qfunc}

\item[\code{qfunc2}] quantile or density functions from \code{params2qfunc}

\item[\code{dist.method}] see \code{DVDtest}

\item[\code{...}] extra arguments in \code{integrate}
\end{ldescription}
\end{Arguments}
%
\begin{Value}
distance functions
\end{Value}
%
\begin{Author}\relax
Meng Xu, Philip Reiss
\end{Author}
\inputencoding{utf8}
\HeaderA{wass.perm}{Calculating the distances via permuted data for each \code{k}}{wass.perm}
%
\begin{Description}\relax
Calculating the distances via permuted data for each \code{k}
\end{Description}
%
\begin{Usage}
\begin{verbatim}
wass.perm(k, vdFun, dat1, dat2, ..., permat, .index, report.every = 10,
  exclude, dist.method)
\end{verbatim}
\end{Usage}
%
\begin{Arguments}
\begin{ldescription}
\item[\code{k}] a scalar, \code{k}th \code{data.frame} of \code{ydata1\&2}

\item[\code{vdFun}] a function, \code{gam} or \code{gamlss}, for fitting the varying distributions

\item[\code{dat1}] \code{k}th \code{data.frame} of \code{ydata1}

\item[\code{dat2}] \code{k}th \code{data.frame} of \code{ydata2}

\item[\code{...}] arguments of \code{vdFun}

\item[\code{permat}] a result of \code{make.perm}

\item[\code{.index}] see \code{eval.index.grid} in \code{DVDtest}

\item[\code{report.every}] a scalar, reporting the number permutation

\item[\code{exclude}] an argument of \code{predict}

\item[\code{dist.method}] see \code{DVDtest}
\end{ldescription}
\end{Arguments}
%
\begin{Value}
a matrix of permuted-data distances
\end{Value}
%
\begin{Author}\relax
Philip Reiss, Meng Xu
\end{Author}
%
\begin{SeeAlso}\relax
\code{wass\_perm}
\end{SeeAlso}
\inputencoding{utf8}
\HeaderA{wass\_perm}{Calculating the distances via permuted data}{wass.Rul.perm}
%
\begin{Description}\relax
Calculating the distances via permuted data
\end{Description}
%
\begin{Usage}
\begin{verbatim}
wass_perm(vdFun, nperm, ydata1, ydata2, eval.index.grid, permat, ...,
  exclude, mc.cores, dist.method)
\end{verbatim}
\end{Usage}
%
\begin{Arguments}
\begin{ldescription}
\item[\code{vdFun}] a function, \code{gam} or \code{gamlss}, for fitting the varying distributions

\item[\code{nperm}] a scalar, number of permutation

\item[\code{ydata1}] see \code{DVDtest}

\item[\code{ydata2}] see \code{DVDtest}

\item[\code{eval.index.grid}] see \code{DVDtest}

\item[\code{permat}] a result of \code{make.perm}

\item[\code{...}] partial arguments of \code{vdFun}

\item[\code{exclude}] an argument of \code{predict}

\item[\code{mc.cores}] a scalar, an argument of \code{mclapply}

\item[\code{dist.method}] see \code{DVDtest}
\end{ldescription}
\end{Arguments}
%
\begin{Value}
an array of distances
\end{Value}
%
\begin{Author}\relax
Meng Xu, Philip Reiss
\end{Author}
%
\begin{SeeAlso}\relax
\code{wass.perm}
\end{SeeAlso}
\printindex{}
\end{document}
